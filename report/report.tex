\documentclass[a4paper]{article}
\usepackage[ngerman]{babel}
\usepackage[utf8]{inputenc}
\usepackage{graphicx}
\usepackage{amsmath}
\usepackage{amssymb}
\usepackage{amsthm}
\usepackage{wasysym}
\usepackage{baskervald}
\usepackage[pdfborder={0 0 0}]{hyperref}
\usepackage[backend=bibtex]{biblatex}

\makeatletter
\hypersetup{
  pdftitle={\@title},
  pdfauthor={\@author},
  pdfsubject={Runen, Ogham-Schrift und das Gotische Alphabet},
  pdfcreator={\LaTeX},
  pdfkeywords={Germanisch, Schriftsystem, Runen, Alphabet, Runische Zeichnungen, Ogham-Schrift, Gotisches Alphabet},
  pdfview=FitV,
  pdflang={de-AT},
  unicode,
  breaklinks,
  bookmarks
}
\makeatother

\author{Lukas Prokop, *}
\title{Runen, Ogham-Schrift und das Gotische Alphabet}
\date{Jan 2015}

\begin{document}
\maketitle
\tableofcontents

\section{Grundlagen}
% * Grundlagen
%  - zeitliche, historische und geografische Einordnung
%  - Skizzierung der Völkerniederlassungen im damaligen Europa?
\subsection{Geschichte}
\subsection{Älteste Funde}

\section{Philologie}
\subsection{Die Schrift}
% * Philologie
%  - Runenschrift als Begriff- und Runenschrift, Schriftrichtung
%  - Überblick über das Alphabet
%  - Ähnlichkeiten zu anderen Schriften?
%  - Germanische Sprachen, Eigenschaften, Geschichte, Verbreitung


\section{Anwendung}
\subsection{Gebrauch bzw. Verwendung}
\subsection{Zusammenhang mit dem Mythischen}
\subsection{Verbreitung der Runen bei den Wikingern/keltischer Kulturkreis}
\subsection{Runenreihen, Runenstein (zB Futhark)}
\subsection{Varianten der Runenschrift}

\section{Ogham-Schrift}
\subsection{Zeitliche und geografische Einordnung}
\subsection{Überblick über das Alphabet}
\subsection{Verwendung}

\section{Gotisch}
\subsection{Zeitliche und geografische Einordnung}
\subsection{Überblick über das Alphabet}
\subsection{Verwendung}

\printbibliography
\end{document}
